\chapterAndLabel{Modifiability Checker}{modifiability-checker}

The Modifiability Checker issues a warning, at compile time, if a program might
throw \code{UnsupportedOperationException} at run time due to invoking a
mutating method (such as \code{add} or \code{remove}) on an unmodifiable
collection.

An \emph{unmodifiable collection} is one that throws
\code{UnsupportedOperationException} if a mutating method is called.
The JDK provides many such collections, for example, the results of:
\begin{itemize}
  \item \code{List.of()}, \code{List.copyOf()}
  \item \code{Set.of()}, \code{Set.copyOf()}
  \item \code{Map.of()}, \code{Map.copyOf()}
  \item \code{Collections.emptyList()}, \code{Collections.emptySet()}, \code{Collections.emptyMap()}
  \item \code{Collections.unmodifiableList()}, \code{Collections.unmodifiableSet()}, \code{Collections.unmodifiableMap()}
\end{itemize}

A \emph{modifiable collection} is one on which mutating methods can be called
without throwing \code{UnsupportedOperationException}.  Many other methods, such
as \code{get}, \code{contains}, \code{isEmpty}, and \code{toArray}, may be
called on any collection, whether modifiable or unmodifiable.

The Modifiability Checker classifies each collection (and iterator) value by its
\emph{modifiability capability}.  It guarantees that mutating methods are only
called on collections that are known to be modifiable.  Thus, it prevents a
common class of runtime bugs where code compiles but fails at execution with
\code{UnsupportedOperationException}.

To run the Modifiability Checker:

\begin{Verbatim}
javac -processor org.checkerframework.checker.modifiability.ModifiabilityChecker MyFile.java
\end{Verbatim}

You can also run it together with other checkers by listing multiple processors
to \code{-processor}.

\sectionAndLabel{Modifiability annotations}{modifiability-annotations}

The Modifiability Checker provides a small lattice of type qualifiers that
describe the modifiability capability of a collection.  The main user-visible
qualifiers are:
\begin{itemize}
  \item \code{@UnknownModifiability},
  \item \code{@Modifiable}, and
  \item \code{@Unmodifiable}.
\end{itemize}
Internally, the checker also uses a bottom qualifier
\code{@BottomModifiable}, which programmers do not write.

The qualifier hierarchy is shown in
Figure~\ref{fig-modifiability-lattice}.

\begin{figure}
  \centering
  \begin{minipage}{0.5\linewidth}
\begin{verbatim}
      @UnknownModifiability
        /             \
@Modifiable         @Unmodifiable
        \              /
        @BottomModifiable
\end{verbatim}
  \end{minipage}
  \caption{Modifiability qualifier hierarchy.}
  \label{fig-modifiability-lattice}
\end{figure}

\begin{description}

\item[\code{@UnknownModifiability}]
The top qualifier in the hierarchy.  The checker does not know whether the
collection is modifiable or unmodifiable; it may be either.  This is the default
qualifier that is applied to unannotated collection types.

\item[\code{@Modifiable}]
The collection is known to support all standard mutating operations, such as
\code{add}, \code{remove}, \code{clear}, and \code{put}, without throwing
\code{UnsupportedOperationException}.  A call to a mutating method is permitted
only when the receiver is annotated as \code{@Modifiable}.

\item[\code{@Unmodifiable}]
The collection is known to be unmodifiable: any mutating operation would throw
\code{UnsupportedOperationException}.  A call to a mutating method on a
\code{@Unmodifiable} receiver always yields a type-checking error.

\item[\code{@BottomModifiable}]
The bottom qualifier in the hierarchy.  It is a subtype of both
\code{@Modifiable} and \code{@Unmodifiable}.  It is used internally by the
checker and is not intended to be written by programmers.

\end{description}

Each annotation’s Javadoc should include a precise description and an example
use, and should link back to this chapter using
\code{@checker\_framework.manual \#modifiability-checker Modifiability Checker}.
Internal qualifiers such as \code{@BottomModifiable} should be documented last.

\sectionAndLabel{What the Modifiability Checker checks}{modifiability-rules}

The Modifiability Checker enforces that mutating methods are only called on
collections that are definitely modifiable.

A \emph{mutating method} is any method that may change the size or contents of a
collection, such as:
\begin{itemize}
  \item \code{add}, \code{addAll}
  \item \code{remove}, \code{removeAll}, \code{retainAll}, \code{clear}
  \item \code{put}, \code{putAll}, \code{replace}, \code{replaceAll}
\end{itemize}

A \emph{read-only method} is any method that does not modify the collection,
such as:
\begin{itemize}
  \item \code{contains}, \code{containsAll}
  \item \code{equals}, \code{hashCode}
  \item \code{get}, \code{getFirst}, \code{getLast}
  \item \code{indexOf}, \code{isEmpty}, \code{toArray}
\end{itemize}

The checker enforces the following rules:

\begin{itemize}
  \item A call to a mutating method is permitted only if the receiver is
        annotated as \code{@Modifiable}.
  \item If the receiver is annotated as \code{@Unmodifiable}, the checker issues
        an error because the call would always throw
        \code{UnsupportedOperationException}.
  \item If the receiver is annotated as \code{@UnknownModifiability}, the checker
        conservatively issues a warning, because the collection might be
        unmodifiable at run time.
  \item Read-only methods may be called on collections of any modifiability
        qualifier (\code{@UnknownModifiability}, \code{@Modifiable},
        \code{@Unmodifiable}) without error.
  \item Subtyping follows the lattice in
        Figure~\ref{fig-modifiability-lattice}: both \code{@Modifiable} and
        \code{@Unmodifiable} are subtypes of \code{@UnknownModifiability}, and
        \code{@BottomModifiable} is a subtype of both.
\end{itemize}

The checker does not rely on run-time tests or exceptions; it reasons entirely
from static types.  Methods that are known to return unmodifiable collections
(such as \code{List.of()} and \code{Collections.unmodifiableList(...)}) should
be annotated in the JDK as returning \code{@Unmodifiable}, so that the checker
can track modifiability through library calls.

\sectionAndLabel{Examples}{modifiability-examples}

\begin{Verbatim}
import java.util.*;

class Demo {

    void factoryExamples() {
        @Modifiable List<String> a = new ArrayList<>();
        @Unmodifiable List<String> b = List.of("x", "y");
        @UnknownModifiability List<String> c = maybeList();

        a.add("z");  // OK: receiver is @Modifiable

        b.add("z");  // error: mutating @Unmodifiable list

        c.add("z");  // error: warning! possible mutation of a collection whose
                     //          modifiability is unknown
    }

    @UnknownModifiability List<String> maybeList() {
        // Might return modifiable or unmodifiable list.
        return Math.random() > 0.5 ? List.of("a") : new ArrayList<>();
    }

    void readonlyExamples(@UnknownModifiability List<String> x,
                          @Unmodifiable List<String> y) {
        x.contains("a"); // OK
        y.get(0);        // OK
        // Read-only methods are safe for any qualifier.
    }
}
\end{Verbatim}